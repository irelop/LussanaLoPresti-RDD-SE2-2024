\section{Purpose}
The \textbf{S\&C system} is a platform designed to connect university students seeking \textbf{internships} with companies offering internship opportunities. The platform helps students and companies connect and use tools designed for their needs.\newline
\textbf{Students} are able to log in to the platform via their university credentials. Once they are logged in, they can \textbf{personalise their profile} by uploading their CV and inserting their skills, experiences and attitudes. They can \textbf{search for an internship} by using the search bar, or by waiting for personalised recommendations sent by the system. If they find a suitable internship, they can \textbf{contact} the company. They might also be contacted by a company, so they can accept or decline the offer. If a company accepts a student's application, the Selection Process begins, which involves an interview.\newline
\textbf{Companies} can publish \textbf{internship advertisements}, look actively for suitable candidates via the search bar or use the recommendations sent by the system. They can, as well, \textbf{contact students}, or they can accept or decline internship requests. They can \textbf{setup an interview} with possible candidates, and create customized forms for use during interviews linked to specific internship advertisements.\newline
Both students and companies can monitor the interactions regarding internships by the \textbf{Monitoring Section}. They can also provide \textbf{feedbacks} on their experiences with one another, whether after an interview or during an internship. These comments are published on the user’s profile.\newline
The \textbf{Recommendation Process} is based on an analysis by the system that considers the criteria used in the search bar by the student, the candidate profile inserted in the internship advertisement, and the feedback provided by both parties.
\subsection{Goals}
\begin{itemize}

    \item [\text{[G1]}] University students can find companies aligned with their interests, apply for internships, complete interviews, and track every stage of the process, from application to the successful completion of the internship.

    \item[\text{[G2]}] Companies can advertise their internship opportunities to find suitable students for the position, contact them, conduct interviews, hire the best candidates, and keep track of the entire process.

    %uniamo g2 e g3? g2 è un goal?
    %aggiungiamo i feedbacks???
    
\end{itemize}

\section{Scope}
\subsection{World Phenomena}

\begin{itemize}

    \item[\text{[WP1]}] University Students want to do an internship.

    \item[\text{[WP2]}] Companies offer internships to university students.

    \item[\text{[WP3]}] Companies are looking for students for internship positions.

    \item[\text{[WP4]}] University Students write their CV with their experiences, skills, and attitudes.

    \item[\text{[WP5]}] Companies decide the project (application domain, tasks to be performed, relevant adopted technologist, etc) and terms offered (salary, benefits, mentorship, etc).

    \item[\text{[WP6]}] Companies and potential candidates (students) establish a contract.

    \item[\text{[WP7]}] Companies do interviews with potential candidates.

    \item[\text{[WP8]}] University Students work as interns in the Companies.
    
\end{itemize}

\subsection{Shared Phenomena}
\subsubsection{World controlled}
\begin{itemize}
    \item [\text{[SP1]}] University Students create a profile and upload their CV.
    \item [\text{[SP2]}] Companies create a profile and the internship position.
    \item [\text{[SP3]}] Companies advertise the internship position.
    \item [\text{[SP4]}] University students look for an internship through the platform.
    \item [\text{[SP5]}] Students contact Companies they are interested in to initiate the process.
    \item [\text{[SP6]}] Companies contact Students who match their requirements to initiate the process.
    \item [\text{[SP7]}] Students and Companies can accept/refuse the offers made to them.
    \item [\text{[SP8]}] University Students and Companies provide feedback to S\&C.
\end{itemize}
\subsubsection{Machine controlled}
\begin{itemize}
    \item [\text{[SP9]}] S\&C platform informs Students when an internship they are interested in becomes available.
    \item [\text{[SP10]}] S\&C platform informs Companies about the availability of Students with CVs corresponding to their needs.
    \item [\text{[SP11]}] S\&C supports the selection process by helping Companies manage interviews and finalize the selection.
    \item [\text{[SP12]}] S\&C provides mechanisms to monitor the matchmaking process for both Students and Companies.
\end{itemize}

\section{Definitions, Acronyms, Abbreviations}
\subsection{Definitions}
\textbf{Single Sign-on} is an authentication method that allows users to log into multiple applications and websites using a single set of credentials. In the S\&C system, Students use their university credentials to log in.\newline
\textbf{Application Programming Interface} is a collection of functions and methods that allow the development of applications to interact with the features or data of an operating system, software, or other services.\newline
\textbf{RESTful APIs}: specific standardised architectural style for APIs.\newline
\textbf{Web Socket}: computer communications protocol, providing a simultaneous two-way communication channel over a single Transmission Control Protocol (TCP) connection.\newline
\textbf{Recommendation} is the process performed by the system that recommends to students potential internship positions that they might be interested in and to companies potential candidates.\newline
The \textbf{Chamber of Commerce Certificate} is an official document provided by the Chamber of Commerce that holds a legal certification value. It confirms the company's registration in the Business Register and includes its name, legal structure, and registration details.\newline
The \textbf{Revenue Agency} is a public entity that is not focused on profit but is dedicated to ensuring optimal tax compliance. Its primary responsibilities include collecting tax revenues, offering services and support to taxpayers, and conducting assessments and inspections to combat tax evasion.
\subsection{Acronyms}
ADV: advertisement\newline
SSO: Single Sign-on \newline
API: Application Programming Interface\newline
REST: Representational State Transfer\newline
HTTPS: HyperText Transfer Protocol Secure\newline
TLS: Transport Layer Security \newline
FCM: Firebase Cloud Messaging \newline
VAT number: value-added tax identification number (\textit {partita IVA})\newline
CEO: Chief Executive Officer \newline
HR: Human Resources\newline
CV: Curriculum Vitae
\subsection{Abbreviations}
G\#: Goal\newline
WP\#: World Phenomena\newline
SP\#: Shared Phenomena\newline
D\#: Domain Assumptions\newline
R\#: Functional Requirement\newline
UC\#: Use Case

\section{Revision History}
Version 1.0: 22/12/2024
Version 2.0: 07/01/2025

\section{Reference Documents}
Specification document: Assignment RDD AY 2024-2025. \newline
Slides of the course "Software Engineering 2" held at Politecnico di Milano by Professor Rossi (a.y. 2024-25).\newline
Graduate Profile Survey 2022 by AlmaLaurea.\newline
Some definitions from the Definitions, Acronyms, Abbreviations section are taken from researches done on the Internet.\newline

\section{Document Structure}
\begin{enumerate}
    \item \textbf{Introduction}: this section includes a description of the system, done through the explanation of the \textbf{Purpose} of the system and the \textbf{Scope} of the problem that includes a list of \textbf{World and Shared Phenomena} concerning the project. There is also technical information to be able to read the document (\textbf{Definitions, Acronyms, Abbreviations}), the \textbf{Revision History}, the \textbf{Reference Documents}, and the \textbf{Document Structure}.
    \item \textbf{Overall Description}: a high-level explanation of how the platform works through the \textbf{Product prospective} section where there are the descriptions of some scenarios, the Domain Class Diagram and the State Diagrams, the \textbf{Product functions} that explain the most important requirements, the \textbf{User Characteristics} section, and the \textbf{Assumption, dependencies, and constraints} section that lists all the Domain Assumptions made to make the project work.
    \item \textbf{Specific Requirements}: detailed analysis of all the aspects shown in Chapter 2 to be used by the development team. This chapter includes the \textbf{External Interface Requirements}, the \textbf{Functional Requirements} explained through the Use Case paradigm, the \textbf{Performance Requirements}, the \textbf{Design Constraints}, and the \textbf{Software System Attributes}.
    \item \textbf{Formal Analysis using Alloy}: modelisation of the problem and formal check done using Alloy 6.
    \item \textbf{Effort Spent}: information about the time spent drafting the document divided per group members.
    \item \textbf{References}: this section lists the documents used to draft the project.
\end{enumerate}